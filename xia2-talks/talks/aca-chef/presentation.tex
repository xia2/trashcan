\documentclass[slides,compress]{beamer}
\usepackage{graphicx,amsmath,hyperref}
\usepackage{verbatim}

\usepackage[normalem]{ulem}

\usetheme{default}
\useinnertheme{rectangles}

\title{\huge Radiation Damage Analysis with Chef}
\subtitle{\large ACA 2012}

\author{Graeme Winter}
\institute{Diamond Light Source}
\date{July 2012}

\begin{document}

\setbeamertemplate{background}{
\includegraphics[width=\paperwidth,height=\paperheight]
{diamond-background.png}
}

\frame{\maketitle}

\frame{
\frametitle{Overview}
\begin{itemize}
\item{Strategy, background, caveats}
\item{Review of statistics}
\item{Example cases}
\item{New statistic - $R_{cp}(d)$}
\item{More example cases}
\end{itemize}
}

\frame{
\frametitle{Strategy, background and caveats}
\begin{itemize}
\item{Radiation damage gives rise to lots of changes}
\item{Will only discuss changes to measured \emph{intensities}}
\item{Assume sufficient data for scaling, analyse after corrections applied}
\item{Make no assumptions about scaling \emph{program}}
\item{Looking to answer the question: would this data set have been
    more useful if I stopped collecting data earlier - balancing:
\begin{itemize}
\item{Gains from additional measurements}
\item{Losses due to systematic differences reducing signal}
\end{itemize}
}
\item{Assume sufficient multiplicity that above question is meaningful}
\end{itemize}
}

\frame{
\frametitle{Not looking for}
\begin{itemize}
\item{Changes to unit cell}
\item{Sample $B$-factor}
\item{Vanishing diffraction}
\end{itemize}
}

\frame{
\frametitle{Looking for}
\begin{itemize}
\item{Changes to measured \emph{intensities}}
\end{itemize}
}

\frame{
\frametitle{$R_{\rm{merge}}$}
\begin{equation}
R_{\rm{merge}} = 
\frac{\sum_{\underline{h}} \sum_j | I_{{\underline{h}}j} - 
\overline{I}_{\underline{h}} |}
{\sum_{\underline{h}} \sum_j I_{{\underline{h}}j}}
\end{equation}
}

\frame{
\frametitle{$R_{\rm{d}}$, Diederichs (2006)}
\begin{equation}
R_d = \frac{\sum_{{\underline{h}}} \sum_{|b_j - b_i| = d}
|I_{\underline{h}j} - I_{\underline{h}i}|}
{\sum_{{\underline{h}}} \sum_{|b_j - b_i| = d}
\frac{1}{2}|I_{\underline{h}j} + I_{\underline{h}i}|}
\end{equation}
}

\frame{
\frametitle{Application to Example Sets}
...
}

\frame{
\frametitle{$R_{cp}(d)$, \emph{Cumulative-Pairwise} Residual}
\begin{equation}
R_{cp}(d) = \frac{\sum_{\lambda} \sum_{\underline{h}} \sum_{
{i \neq j}, i:d_i \leq d, j:d_j \leq d} |I_{\underline{h}i} - 
I_{\underline{h}j}|}
{\sum_{\lambda} \sum_{\underline{h}} 
\sum_{i \neq j, i:d_i \leq d, j:d_j \leq d} \frac{1}{2}|I_{\underline{h}i} + 
I_{\underline{h}j}|}
\end{equation}
}


\end{document}
