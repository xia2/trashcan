\documentclass[slides,compress]{beamer}
\usepackage{graphicx,amsmath,hyperref}
\usepackage{verbatim}

\usepackage[normalem]{ulem}

\usetheme{default}
\useinnertheme{rectangles}

\title{\huge Novel Uses for \emph{xia2} in a Beamline Environment}
\subtitle{\large Harvard Medical School, July 2012}

\author{Graeme Winter}
\institute{Diamond Light Source}
\date{July 2012}

\begin{document}

\setbeamertemplate{background}{
\includegraphics[width=\paperwidth,height=\paperheight]
{diamond-background.png}
}

\frame{\maketitle}

\frame{
\frametitle{Overview}
\begin{itemize}
\item{Context}
\item{At Diamond Light Source}
\item{Options for lower level interaction}
\item{How does this help?}
\item{Discussion, acknowledgements}
\end{itemize}
}

\frame{
\frametitle{Context / Background}
\begin{itemize}
\item{\emph{xia2} developed as part of e-HTPX project}
\item{Started work on DNA project in 2002 - kept this in mind}
\item{Original thoughts behind xia2 were to cover scope from data
    collection through reduction to phasing / refinement}
\item{Implemented middle bit}
\end{itemize}
}

\frame{
\frametitle{\emph{xia2} at Diamond Light Source}
\begin{itemize}
\item{Full functionality available to users}
\item{Run automatically (has been so for five years or so) on a
    per-sweep basis - xia2 -blah -image /path/to/an/image}
\item{Automatic running essentially fire and forget - results
    available to user, not integrated into data collection system}
\item{Rather than providing fine grained user control, run several
    jobs}
\item{Inspired development of fast DP - a very much cut down
    ``script'' for performing simple data analysis with XDS}
\end{itemize}
}

\frame{
\frametitle{Issues with this approach}
\begin{itemize}
\item{Lack of user interaction}
\item{Does not sit nicely with e.g. MAD, multi-sweep data collection
    (which we have plans for)}
\item{Does not give feedback directly to the user (which we can mend,
    and are working on)}
\item{\emph{Does not handle incremental feedback well}}
\end{itemize}
}

\frame{
\frametitle{Lower-level interaction}
\begin{itemize}
\item{Main \emph{xia2} program is a print statement (really) - all
    processing performed to deliver the results of this print
    statement}
\item{However, the code is \emph{just Python}}
\item{Let's look at a lower level interaction}
\end{itemize}
}

\begin{frame}[fragile]
\tiny{
\begin{verbatim}
    from Handlers.PipelineSelection import add_preference
    directory, image = os.path.split(sys.argv[1])

    from XProject import XProject
    from XCrystal import XCrystal
    from XWavelength import XWavelength
   
    xp = XProject(name = 'example')
    xc = XCrystal('demonstration', xp)
    xw = XWavelength('native', xc)
    xw.add_sweep('native', directory, image)
   
    print 'Scaled data: %s' % xc.get_scaled_merged_reflections()['mtz']
\end{verbatim}
}
\end{frame}

\frame{
\frametitle{Observations}
\begin{itemize}
\item{Code above will work}
\item{It will also write a lot of junk to stdout}
\item{May also generate extra directories - full fat version next slide}
\end{itemize}
}

\begin{frame}[fragile]
\tiny{
\begin{verbatim}
    from Handlers.PipelineSelection import add_preference
    from Handlers.Streams import streams_off
    from Handlers.Environment import Environment

    Environment.dont_setup()
    streams_off()

    add_preference('indexer', 'labelit')
    add_preference('integrater', 'xdsr')
    add_preference('scaler', 'xdsr')

    directory, image = os.path.split(sys.argv[1])

    from XProject import XProject
    from XCrystal import XCrystal
    from XWavelength import XWavelength
   
    xp = XProject(name = 'example')
    xc = XCrystal('demonstration', xp)
    xw = XWavelength('native', xc)
    xw.add_sweep('native', directory, image)
   
    print 'Scaled data: %s' % xc.get_scaled_merged_reflections()['mtz']
\end{verbatim}
}
\end{frame}

\frame{
\frametitle{Moving on}
\begin{itemize}
\item{This code will \emph{silently} process your data, scale and
    return a merged MTZ file containing intensities and amplitudes}
\item{Provided handle to xp or xc kept can interrogate nearly everything}
\item{\emph{Aha!} why can't I just add another sweep and get
    reflections again? Bugs}
\item{Code was never written to work like this, though with a couple
    of hours work could be}
\end{itemize}
}

\frame{
\frametitle{How does this help?}
\begin{itemize}
\item{Lower level access to \emph{xia2} machinery}
\item{More control from caller pespective}
\item{Capability to provide user interface - user can (in principle)
    add and remove sweeps, tweak processing on live system}
\item{Capability for system to keep adding sweeps (say multi-crystal
    environment) until complete data set achieved}
\end{itemize}
}

\frame{
\frametitle{What needs doing?}
\begin{itemize}
\item{Mainly resolving dependencies in the analysis - xia2 is dynamic}
\item{e.g. when sweep added to wavelength, scaling needs repeating -
    when images added to a sweep integration needs repeating}
\item{Probably a couple of days work}
\item{Oh, and moving all command options to Phil parameters}
\end{itemize}
}

\frame{
\frametitle{Discussion, acknowledgements}
\begin{itemize}
\item{Kept talk short so can show code}
\item{Thanks to Nick Sauter for suggesting this - moving in an
    interesting direction}
\item{\emph{xia2} funded through BBSRC e-Science e-HTPX project, CCP4,
    BioXHit and Diamond Light Source}
\item{Users, providers of test data, lots of people who have provided
    input into project}
\end{itemize}
}

\end{document}
