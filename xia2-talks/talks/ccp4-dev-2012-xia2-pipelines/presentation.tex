\documentclass[slides,compress]{beamer}
\usepackage{graphicx,amsmath,hyperref}
\usepackage{verbatim}

\usepackage[normalem]{ulem}

\usetheme{default}
\useinnertheme{rectangles}

\title{\huge xia2 \& pipelines}
\subtitle{\large CCP4 developers meeting 2012}

\author{Graeme Winter}
\institute{Diamond Light Source}
\date{21 March 2012}

\begin{document}

\setbeamertemplate{background}{
\includegraphics[width=\paperwidth,height=\paperheight]
{diamond-background.png}
}

\frame{\maketitle}

\frame{
\frametitle{Overview}
\begin{itemize}
\item{xia2 developments}
\item{Removal of frame count limitations}
\item{Current state of multi-crystal analysis}
\item{Plans for multi-crystal analysis}
\item{Fast EP}
\end{itemize}
}

\frame{
\frametitle{xia2 developments}
\begin{itemize}
\item{xia2 for multi-axis goniometers and small unit cells}
\item{Bugs / fixes}
\item{Using Aimless}
\item{\sout{P1 integration}}
\item{\sout{Scaling XDS data with Scala / Aimless}}
\end{itemize}
}

\frame{
\frametitle{Removal of frame count limitations}
\begin{itemize}
\item{Use Aimless in place of Scala}
\item{Use Pointless in place of Reindex\footnote{This is still not as
      efficient as it could be - but it works}}
\item{Run -3daii or -3da}
\item{No more batch limitations}
\end{itemize}
}

\frame{
\frametitle{Plate data}
\begin{itemize}
\item{Data from Diamond I04-1}
\item{110 x 100 image sweeps (20 degree)}
\item{Processed with xia2 -3daii -failover -spacegroup ... -cell ...}
\item{... crashed (this is to be expected!)}
\item{In XSCALE output analyse $CC(i,j)$, scale to $d = \frac{1}{CC} -
    1$ - useful distances}
\item{Get dendogram}
\end{itemize}
} 

\frame{
\frametitle{Results}
\begin{center}
\includegraphics[scale=0.45]{L1.pdf}
\end{center}
}

\frame{
\frametitle{Next steps}
\begin{itemize}
\item{Remove useless sweeps, run again}
\item{This process is partially automated}
\item{xia2 -3daii (etc.) -xinfo mod.xinfo}
\item{Runs now to completion}
\end{itemize}
}

\begin{frame}[fragile]
\begin{verbatim}
For AUTOMATIC/DEFAULT/NATIVE
High resolution limit     1.77    7.93    1.77
Low resolution limit     39.42   39.42    1.82
Completeness             88.7    99.5    19.7
Multiplicity             39.2    46.8     1.8
I/sigma                  21.0    35.1     3.1
Rmerge                  0.275   0.261   0.204
Rmeas(I)                0.279   0.264   0.281
Rmeas(I+/-)             0.280   0.265   0.268
Rpim(I)                 0.037   0.039   0.168
Rpim(I+/-)              0.049   0.047   0.172
Wilson B factor         22.655
Total observations      429848  8325    318
Total unique            10962   178     174
\end{verbatim}
\end{frame}

\frame{
\frametitle{Results}
\begin{center}
\includegraphics[scale=0.45]{L2.pdf}
\end{center}
}

\frame{
\frametitle{Cluster 1: 37, 38, 16, 39, 8, 9, 5, 6, 3, 12}
\begin{itemize}
\item{Use Re-Xinfo to take selected runs from dendogram, x1335 output
    and automatic.xinfo to generate new xinfo}
\item{... and run \emph{again}}
\end{itemize}
}

\begin{frame}[fragile]
\begin{verbatim}
High resolution limit     1.74    7.79    1.74
Low resolution limit     39.42   39.42    1.79
Completeness             84.2    93.4    16.3
Multiplicity             11.2    12.3     1.5
I/sigma                  17.4    31.9     2.5
Rmerge                  0.099   0.068   0.213
Rmeas(I)                0.105   0.072   0.334
Rmeas(I+/-)             0.105   0.072   0.292
Rpim(I)                 0.026   0.019   0.211
Rpim(I+/-)              0.034   0.022   0.199
Wilson B factor         22.448
Total observations      121810  2097    237
Total unique            10858   170     153
\end{verbatim}
\end{frame}

\frame{
\frametitle{Results}
\begin{center}
\includegraphics[scale=0.45]{cluster1.pdf}
\end{center}
}

\frame{
\frametitle{Results}
\begin{center}
\includegraphics[scale=0.35]{Rmerge.pdf}
\end{center}
}

\frame{
\frametitle{Next steps: April 2012 onwards}
\begin{itemize}
\item{Automate this procedure, allow user to specify requirements}
\item{Include BLEND analysis: complementary information}
\item{Code up own $CC$ calculation: can run from Aimless output}
\item{Automate $R_{\rm{merge}}$ \emph{vs.} BATCH \emph{vs.} run
    analysis}
\item{Work on simply processing more tricky / weaker data}
\end{itemize}
}

\frame{
\frametitle{Fast EP}
\begin{itemize}
\item{Experimental phasing in the spirit of Fast DP}
\item{Uses shelxc / d\_mp / e}
\item{Extentive use of CCTBX for analysis}
\item{Can give maps within 2 minutes of experiment in favourable cases
    - DP 1 minute, EP 1 minute}
\item{Takes no information, searches everything (spacegroups, sites, hands)}
\end{itemize}
}

\frame{
\frametitle{Brute force / power}
\begin{itemize}
\item{Assess with CCTBX code}
\item{Use shelxc to get ins file, hkl file}
\item{Modify using CCTBX}
\item{Run shelxd\_mp on cluster: up to 480 x 3GHz cpu's}
\item{Sort results on CFOM, find best, get number sites}
\item{Run 22 shelxe runs: 2 hands, 11 solvent fractions, score on
    pseudo-free $CC$}
\end{itemize}
}

\frame{
\frametitle{Testing / proof of principle}
\begin{itemize}
\item{Data from Raj / Pavel}
\item{Trivial, tricky, impossible examples}
\item{Immediate response: phasing quality in shelxe depends on
    resolution}
\item{Substructure solution much less limited}
\end{itemize}
}

\frame{
\frametitle{Limitations}
\begin{itemize}
\item{Shelxe limited for phasing: use something more potent?}
\item{Hand determination}
\end{itemize}
}

\frame{
\frametitle{Value}
\begin{itemize}
\item{Fast feedback for phasing experiments}
\item{Solve structures autmatically for users}
\item{Has already cracked a useful novel structure}
\end{itemize}
}

\end{document}
